
\documentclass[11pt,a4paper]{report}

\usepackage[portuges]{babel}
\usepackage[utf8]{inputenc} % define o encoding usado texto fonte (input)--usual "utf8" ou "latin1
\usepackage{graphicx} %permite incluir graficos, tabelas, figuras
\usepackage{subcaption}
\usepackage{listings}
\usepackage{color}
\usepackage{multicol}
\usepackage{indentfirst}
\usepackage{hyperref}
\usepackage{amsmath}
\usepackage{amssymb}
\usepackage{float}
\usepackage[inline]{enumitem}

\definecolor{mygreen}{rgb}{0,0.6,0}
\definecolor{mygray}{rgb}{0.5,0.5,0.5}
\definecolor{mymauve}{rgb}{0.58,0,0.82}

\lstset{ %
  backgroundcolor=\color{white},   % choose the background color
  basicstyle=\footnotesize,        % size of fonts used for the code
  breaklines=true,                 % automatic line breaking only at whitespace
  captionpos=b,                    % sets the caption-position to bottom
  commentstyle=\color{mygreen},    % comment style
  escapeinside={\%*}{*)},          % if you want to add LaTeX within your code
  keywordstyle=\color{blue},       % keyword style
  stringstyle=\color{mymauve},     % string literal style
}

\title{Processamento de Linguagens e Compiladores - Trabalho Prático 1\\
       \textbf{Grupo 3}\\ Relatório
       } %Titulo do documento
%\title{Um Exemplo de Artigo em \LaTeX}
\author{Alef Pinto Keuffer\\ (A91383)\and Catarina Martins Sá Quintas\\ (A91650)\and  Ivo Miguel Gomes Lima \\(A90214)
       } %autores do documento
\date{\today} %data

\begin{document}
	\begin{minipage}{0.9\linewidth}
        \centering
		\includegraphics[width=0.4\textwidth]{um.jpeg}\par\vspace{1cm}
                \href{https://www.uminho.pt/PT}
		{\scshape\LARGE Universidade do Minho} \par
		\vspace{0.6cm}
                \href{https://lcc.di.uminho.pt}
		{\scshape\Large Licenciatura em Ciências da Computação} \par
		\maketitle
	\end{minipage}

\tableofcontents % insere Indice

\chapter{Introdução}

No âmbito da unidade curricular Processamento de Linguagens e compiladores (PLC),
foi-nos proposto a realização de um projeto de forma a aprofundar os nossos
conhecimentos adquiridos na sala de aula, atingindo os seguintes objetivos:

\begin{itemize}
    \item Aumentar a capacidade de escrever Expressões Regulares(ER)
    \item Desenvolver sistematicamente Processadores de Linguagens Regulares, ou Filtros de Texto.
    \item Familiarizar com o módulo 're' presente no Python.
\end{itemize}
  

Para o efeito, criamos um  processador de Bib\TeX \href{http://www.bibtex.org/}. 
A Bib\TeX \ é uma ferramenta de formatação usada em documentos em La\TeX.
Um exemplo desta ferramenta: 
\begin{lstlisting}
@techreport{jspell1,
   author = "J.J. Almeida and Ulisses Pinto",
   title = "Manual de Utilizador do {JSpell}",
   year = 1994,
   type = "Manual",
   month = "Jul",
   institution = "umdi",
   keyword = "morphology, lexical analysis,jspell",
   abstract = {},
   url = "http://natura.di.uminho.pt/~jj/pln/jspellman.ps.gz",
}

\end{lstlisting}

Existe um conjunto de campos obrigatórios e facultativos para que um Bib\TeX \ seja válido, alguns desses campos são: \emph{article}, \emph{book},\emph{inproceedings},\emph{misc},\emph{proceedings} entre outros.



\newpage

\section{Descrição do Problema}
\newcommand{\gv}{\emph{GraphViz}}
\newcommand{\htlm}{\emph{HTML}}
\newcommand{\dott}{\emph{Dot}}
\newcommand{\bib}{Bib\TeX}



A nossa solução deve satisfazer os seguintes requisitos: 
\begin{enumerate}[label=R\arabic*]
\item\label{R1} Fazer a contagem das categorias presentes no documento, tais como: \emph{phDThesis}, \emph{Misc}, \emph{InProceeding }, \emph{etc }.
\item\label{R2}Produzir um documento em formato \htlm \ com
\begin{enumerate*}[label=(R2.\arabic*)]
  \item\label{R21} o nome das categorias encontradas e
  \item\label{R22} respectivas contagens.
\end{enumerate*}
\item\label{R3} Filtrar, para cada entrada de cada categoria, a respetiva
\begin{enumerate*}[label=(R3.\arabic*)]
  \item\label{R31} chave
  \item\label{R32} autores,
  \item\label{R33} e título.
  \item\label{R34} O resultado final deverá ser incluído no documento \htlm \ gerado \ref{R2}.
\end{enumerate*}
\item\label{R4} Criar um índice de autores, que mapeie cada autor nos respectivos registos, de modo a que posteriormente uma ferramenta de procura do Linux possa fazer a pesquisa.
\item\label{R5} Construir um Grafo que mostre, para um dado autor (definido à partida) todos os autores que publicam normalmente com o autor em causa.
\item\label{R6}Recorrendo à linguagem \dott \  do \gv, gerar um ficheiro com o grafo de \ref{R5} de modo a que possa, posteriormente, usar uma das ferramentas que processam \dott \  para desenhar o dito grafo de associações de autores.
\end{enumerate}


\section{Estratégia de Resolução}

Nossa estratégia consiste em associar pares de (tipo da publicação, chave) com os campos da entrada e aplicar diversas manipulações sobre essa estrutura para satisfazer os requisitos.

A estratégia para satisfazer \ref{R1} consistiu em ler o arquivo ler linha a linha verificando se a categoria encontrada ja ocorreu no dicionário, se esta já existir, irá ser incrementada, senão será guardada como sendo a primeira ocorrência. 

produzindo no final um ficheiro \htlm \ tendo-se criado uma lista de strings 
\section{Implementação}


\begin{lstlisting}[language=python]


import re

def main():
    
    file = open("exemplo-utf8.bib", "r")
    read = True
    dic = {}
    string_ls = ['<!DOCTYPE  HTML PUBLIC>\n<HTML>\n   <HEAD>\n      <TITLE>Categories in BibTeX</TITLE>\n </HEAD>\n   <BODY>']
    while read:
      linhaFicheiro = file.readline()
      ncat = re.match(r'^@(.*){',linhaFicheiro)
      if ncat != None:
        cat_title = ncat.group(1).title()
        dic[cat_title] = dic.get(cat_title,0) + 1
      if not linhaFicheiro:
        read = False
    file.close()

    time = lambda v: 's' if v > 1 else ''

    for k, v in dic.items():
      string_ls.append(f'      <P>The category {k} appears {v} time{time(v)}.</P>')
    string_ls.append(f'   </BODY>\n{prog.test_data_view()}</HTML>')

    with open('output.html','w') as file:
        file.write('\n'.join(string_ls))
\end{lstlisting}

\chapter{Exercício 4 b}

\subsection{Descrição do Problema}


\section{Estratégia/Implementação de Resolução}


\chapter{Exercício 4 c}

\subsection{Descrição do Problema}


\section{Estratégia/Implementação de Resolução}


\chapter{Exercício 4 d}

\subsection{Descrição do Problema}


\section{Estratégia/Implementação de Resolução}


\chapter{Conclusão}


\end{document}
