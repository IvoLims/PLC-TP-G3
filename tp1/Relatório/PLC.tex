
\documentclass[11pt,a4paper]{report}

\usepackage[portuges]{babel}
\usepackage[utf8]{inputenc} % define o encoding usado texto fonte (input)--usual "utf8" ou "latin1
\usepackage{graphicx} %permite incluir graficos, tabelas, figuras
\usepackage{subcaption}
\usepackage{listings}
\usepackage{color}
\usepackage{multicol}
\usepackage{indentfirst}
\usepackage{hyperref}
\usepackage{amsmath}
\usepackage{amssymb}
\usepackage{float}
\usepackage{enumitem}

\definecolor{mygreen}{rgb}{0,0.6,0}
\definecolor{mygray}{rgb}{0.5,0.5,0.5}
\definecolor{mymauve}{rgb}{0.58,0,0.82}

\lstset{ %
  backgroundcolor=\color{white},   % choose the background color
  basicstyle=\footnotesize,        % size of fonts used for the code
  breaklines=true,                 % automatic line breaking only at whitespace
  captionpos=b,                    % sets the caption-position to bottom
  commentstyle=\color{mygreen},    % comment style
  escapeinside={\%*}{*)},          % if you want to add LaTeX within your code
  keywordstyle=\color{blue},       % keyword style
  stringstyle=\color{mymauve},     % string literal style
}

\title{Processamento de Linguagens e Compiladores - Trabalho Prático 1\\
       \textbf{Grupo 3}\\ Relatório
       } %Titulo do documento
%\title{Um Exemplo de Artigo em \LaTeX}
\author{Alef Pinto Keuffer\\ (A91383)\and Catarina Martins Sá Quintas\\ (A91650)\and  Ivo Miguel Gomes Lima \\(A90214)
       } %autores do documento
\date{\today} %data

\begin{document}
	\begin{minipage}{0.9\linewidth}
        \centering
		\includegraphics[width=0.4\textwidth]{um.jpeg}\par\vspace{1cm}
                \href{https://www.uminho.pt/PT}
		{\scshape\LARGE Universidade do Minho} \par
		\vspace{0.6cm}
                \href{https://lcc.di.uminho.pt}
		{\scshape\Large Licenciatura em Ciências da Computação} \par
		\maketitle
	\end{minipage}

\tableofcontents % insere Indice

\chapter{Introdução}


\chapter{Exercício 4 a}

\subsection{Descrição do Problema}


\section{Estratégia/Implementação de Resolução}

\begin{lstlisting}[language=python]
import re

def main():
    # Funcao do Ivo, nao e esboco
    file = open("exemplo-utf8.bib", "r")
    read = True
    dic = {}
    string_ls = ['<!DOCTYPE  HTML PUBLIC>\n<HTML>\n   <HEAD>\n      <TITLE>Categories in BibTeX</TITLE>\n <script type="text/x-mathjax-config"> MathJax.Hub.Config({"extensions":["tex2jax.js"],"jax":["input/TeX","output/HTML-CSS"],"messageStyle":"none","tex2jax":{"processEnvironments":false,"processEscapes":true,"inlineMath":[["$","$"],["\\(","\\)"]],"displayMath":[["$$","$$"],["\\[","\\]"]]},"TeX":{"extensions":["AMSmath.js","AMSsymbols.js","noErrors.js","noUndefined.js"]},"HTML-CSS":{"availableFonts":["TeX"]}}); </script> <script type="text/javascript" async src="file:////home/useralef/.vscode/extensions/shd101wyy.markdown-preview-enhanced-0.6.1/node_modules/@shd101wyy/mume/dependencies/mathjax/MathJax.js" charset="UTF-8"></script>  </HEAD>\n   <BODY>']
    while read:
      linhaFicheiro = file.readline()
      ncat = re.match(r'^@(.*){',linhaFicheiro)
      if ncat != None:
        cat_title = ncat.group(1).title()
        dic[cat_title] = dic.get(cat_title,0) + 1
      if not linhaFicheiro:
        read = False
    file.close()

    time = lambda v: 's' if v > 1 else ''

    for k, v in dic.items():
      string_ls.append(f'      <P>The category {k} appears {v} time{time(v)}.</P>')
    string_ls.append(f'   </BODY>\n{prog.test_data_view()}</HTML>')

    with open('output.html','w') as file:
        file.write('\n'.join(string_ls))
\end{lstlisting}

\chapter{Exercício 4 b}

\subsection{Descrição do Problema}


\section{Estratégia/Implementação de Resolução}


\chapter{Exercício 4 c}

\subsection{Descrição do Problema}


\section{Estratégia/Implementação de Resolução}


\chapter{Exercício 4 d}

\subsection{Descrição do Problema}


\section{Estratégia/Implementação de Resolução}


\chapter{Conclusão}


\end{document}